\section{Overview}
Control unit is responsible for generating the control signals that are used to activate several operations throughout the pipeline. Also, it's responsible for the extraction of specific information from instruction bits.

\begin{itemize}
    \item It communicates with:
    \begin{itemize}
        \item IF/ID buffer: for reading the instruction bits.
        \item ID/EX buffer: for writing the appropriate registers, ALUop and signals.
        \item Register file: for selecting the registers needed to be read (Rsrc1 and Rsrc2).
        \item Hazards units (HDU and Branch Address Unit): for sending enables and needed signals.
    \end{itemize}
    
    \item Unit Interface:
    \begin{itemize}
    
        \item Inputs: 
        \begin{itemize}
            \item Instruction Bits (32 bits)
            \item Interrupt bit (1 bit)
        \end{itemize}
        
        \item Outputs:
        \begin{itemize}
            \item Rsrc2\_val (32 bits) for immediate values or effective addresses
            \item Rsrc1\_sel (4 bits)
            \item Rsrc2\_sel (4 bits)
            \item Rdst1\_sel (4 bits)
            \item Rdst2\_sel (4 bits) used only in case of swap
            \item Branch/IO Enable (2 bits)
            \item OP2\_sel (1 bit)
            \item SP Enable (1 bit)
            \item OpCode (7 bits)
            \item Branch Enable (1 bit)
            \item ALUop (4 bits)
            \item R/W Control Signal (2 bits)
        \end{itemize}
        
    \end{itemize}
    
    \item Interpretation:
    \begin{itemize}
        \item \textbf{Rsrc2\_val (32 bits):} occupies a single place in the ID/EX buffer. However, it's used in many different ways. It can be used as a register value extracted form register file. it can be used as an immediate value extracted from IF/ID buffer. Also, it can hold the stack pointer address, as well as the effective address (EA) sent to the memory for reading or writing.
        \item \textbf{Rdst2 (4 bits):} only used when dealing with a SWAP instruction, thus we need Op1 and Op2 and their new selectors.
        \item \textbf{OP2\_sel (1 bit):} determines the value of Rsrc2 register in ID/EX buffer, whether it's immediate or register value.
        \item  \textbf{Branch/IO Enable (2 bits):} informs the register file what operation of these are we executing (No/In/Out/Branch), however \emph{Branch Enable} (1 bit) interacts with the Branch Address Unit, informing it what type of OpCode are we dealing with (branching or not).
    \end{itemize}
\end{itemize}

\section{Control Signals}
In this section, instructions are divided into seven types based on the signals produced:
\begin{itemize}
    \item One Operand (not,inc,dec,out,in).
    \item Two Operands (add,sub,and,or).
    \item Immediate Operand (iadd,shl,shr,ldm).
    \item Data (ldd,std).
    \item Stack (push,pop,call,ret,rti).
    \item Jump (jz, jmp).
    \item Special (nop,swap,reset,int).
\end{itemize}   

\subsection{One Operand Instructions}
\begin{itemize}
    \item IB[31:0] are the instruction bits.
    \item Inserting (1111) to Rsrc/Rdst selectors informs the register file not to output any register values.
    \item 'x' indicates don't care.
    \item 0000 at the ALUop indicates no operation.
    \item Rsrc1\_sel is the same as Rdst1\_sel.
\end{itemize}

\begin{center}
    \captionof{table}{One Operand Instruction Control Signals Part I\label{tab:1op1}}
\begin{tabular}{||p{20mm}| p{15mm}| p{15mm}| p{15mm}| p{15mm}| p{15mm}| p{15mm}||} 
\hline
Instruction & OPCode & ALUop & Rsrc1 selector & Rsrc2 selector & Rdst1 selector & Rsrc2 value \\ [0.5ex] 
\hline\hline
NOT & IB[31:25] & 0111 & 0 and IB[24:22] & 1111 & 0 and IB[24:22] & x  \\
\hline
INC & IB[31:25] & 0001 & 0 and IB[24:22] & 1111 & 0 and IB[24:22] & x \\
\hline
DEC & IB[31:25] & 0010 & 0 and IB[24:22] & 1111 & 0 and IB[24:22] & x \\
\hline
OUT & IB[31:25] & 0000 & 0 and IB[24:22] & 1111 & 0 and IB[24:22] & x \\
\hline
IN  & IB[31:25] & 0000 & 0 and IB[24:22] & 1111 & 0 and IB[24:22] & x \\
\hline
\end{tabular}
\end{center}

\begin{center}
    \captionof{table}{One Operand Instruction Control Signals Part II\label{tab:1op2}}
\begin{tabular}{||p{20mm}| p{15mm}| p{15mm}| p{15mm}| p{15mm}| p{15mm}| p{15mm}||} 
\hline
Instruction & OP2 selector & Rdst2 (swap) & Branch /IO Enable & SP Enable & Branch Enable & R/W Control Signal \\ [0.5ex] 
\hline\hline
NOT & x & 1111 & 00 & 0 & 0 & 00 \\
\hline
INC & x & 1111 & 00 & 0 & 0 & 00 \\
\hline
DEC & x & 1111 & 00 & 0 & 0 & 00 \\
\hline
OUT & x & 1111 & 01 & 0 & 0 & 00 \\
\hline
IN  & x & 1111 & 10 & 0 & 0 & 00 \\
\hline
\end{tabular}
\end{center}

\subsection{Two Operand Instructions}
\begin{itemize}
    \item OP2\_sel: 0 the register value and 1 the imm/ea value.
\end{itemize}

\begin{center}
    \captionof{table}{Two Operands Instruction Control Signals Part I\label{tab:2op1}}
\begin{tabular}{||p{20mm}| p{15mm}| p{15mm}| p{15mm}| p{15mm}| p{15mm}| p{15mm}||} 
\hline
Instruction & OPCode & ALUop & Rsrc1 selector & Rsrc2 selector & Rdst1 selector & Rsrc2 value  \\ [0.5ex] 
\hline\hline
ADD & IB[31:25] & 0011 & 0 and IB[27:25] & 0 and IB[24:22] & 0 and IB[21:19] & x \\
\hline
SUB & IB[31:25] & 0100 & 0 and IB[27:25] & 0 and IB[24:22] & 0 and IB[21:19] & x \\
\hline
AND & IB[31:25] & 0101 & 0 and IB[27:25] & 0 and IB[24:22] & 0 and IB[21:19] & x \\
\hline
OR  & IB[31:25] & 0110 & 0 and IB[27:25] & 0 and IB[24:22] & 0 and IB[21:19] & x \\
\hline
\end{tabular}
\end{center}

\begin{center}
    \captionof{table}{Two Operands Instruction Control Signals Part II\label{tab:2op2}}
\begin{tabular}{||p{20mm}| p{15mm}| p{15mm}| p{15mm}| p{15mm}| p{15mm}| p{15mm}||} 
\hline
Instruction & OP2 selector & Rdst2 (swap) & Branch /IO Enable & SP Enable & Branch Enable & R/W Control Signal  \\ [0.5ex] 
\hline\hline
ADD & 0 & 1111 & 00 & 0 & 0 & 00 \\
\hline
SUB & 0 & 1111 & 00 & 0 & 0 & 00 \\
\hline
AND & 0 & 1111 & 00 & 0 & 0 & 00 \\
\hline
OR  & 0 & 1111 & 00 & 0 & 0 & 00 \\
\hline
\end{tabular}
\end{center}

\subsection{Immediate Operand Instructions}
\begin{itemize}
    % Notes
    \item Rsrc1\_sel is the same as Rdst1\_sel, in SHL and SHR cases. However, in IADD case, it's a different register and in LDM case, there's no need for Rsrc, it's just a destination.
    \item In IADD case, Rsrc != Rdst.
    \item In LDM case, there's no Rsrc, it's Rdst.
    \item Rsrc2\_val is the immediate value extracted from the IF/ID buffer.
    \item R/W memory (11) is write and (10) is read.
    \item Sign extend unit is used to adjust the (16 bits) immediate value to (32 bits).
    \item SE: sign extend enable (0/1).
\end{itemize}

\begin{center}
    \captionof{table}{Immediate Operand Instruction Control Signals Part I\label{tab:immop1}}
\begin{tabular}{||p{20mm}| p{15mm}| p{15mm}| p{15mm}| p{15mm}| p{15mm}| p{15mm}||} 
\hline
Instruction & OPCode & ALUop & Rsrc1 selector & Rsrc2 selector & Rdst1 selector & Rsrc2 value  \\ [0.5ex] 
\hline\hline
IADD& IB[31:25] & 0011 & 0 and IB[27:25] & 1111 & 0 and IB[24:22] & 0XSE and IB[15:0] \\
\hline
SHL & IB[31:25] & 1000 & 0 and IB[27:25] & 1111 & 0 and IB[27:25] & 0XSE and IB[15:0] \\
\hline
SHR & IB[31:25] & 1001 & 0 and IB[27:25] & 1111 & 0 and IB[27:25] & 0XSE and IB[15:0] \\
\hline
LDM & IB[31:25] & 0000 & 1111 & 1111 & 0 and IB[27:25] & 0XSE and IB[15:0] \\
\hline
\end{tabular}
\end{center}

\begin{center}
    \captionof{table}{Immediate Operand Instruction Control Signals Part II\label{tab:immop2}}
\begin{tabular}{||p{20mm}| p{15mm}| p{15mm}| p{15mm}| p{15mm}| p{15mm}| p{15mm}||} 
\hline
Instruction & OP2 selector & Rdst2 (swap) & Branch /IO Enable & SP Enable & Branch Enable & R/W Control Signal  \\ [0.5ex] 
\hline\hline
IADD & 1 & 1111 & 00 & 0 & 0 & 00  \\
\hline
SHL & 1 & 1111 & 00 & 0 & 0 & 00 \\
\hline
SHR & 1 & 1111 & 00 & 0 & 0 & 00 \\
\hline
LDM & 1 & 1111 & 00 & 0 & 0 & 11 \\
\hline
\end{tabular}
\end{center}

\subsection{Data Instructions}
Note that:
\begin{itemize}
    % Notes
    \item Effective address does not need a sign extend, that's why it's always zero extended with only 12 bits.
    \item OP2\_sel is 1 to pass the EA.
    \item R/W memory (11) is write and (10) is read.
\end{itemize}

\begin{center}
    \captionof{table}{Data Instruction Control Signals Part I\label{tab:dop1}}
\begin{tabular}{||p{20mm}| p{15mm}| p{15mm}| p{15mm}| p{15mm}| p{15mm}| p{15mm}||} 
\hline
Instruction & OPCode & ALUop & Rsrc1 selector & Rsrc2 selector & Rdst1 selector & Rsrc2 val \\ [0.5ex] 
\hline\hline
LDD & IB[31:25] & 0000 & 0 and IB[27:25] & 1111 & 1111 & 0x000 and IB[19:0] \\
\hline
STD & IB[31:25] & 0000 & 1111 & 1111 & 0 and IB[27:25] & 0x000 and IB[19:0] \\
\hline
\end{tabular}
\end{center}

\begin{center}
    \captionof{table}{Data Instruction Control Signals Part II\label{tab:dop2}}
\begin{tabular}{||p{20mm}| p{15mm}| p{15mm}| p{15mm}| p{15mm}| p{15mm}| p{15mm}||} 
\hline
Instruction & OP2 selector & Rdst2 (swap) & Branch /IO Enable & SP Enable & Branch Enable & R/W Control Signal  \\ [0.5ex] 
\hline\hline
LDD & 1 & 1111 & 00 & 0 & 0 & 10 \\
\hline
STD & 1 & 1111 & 00 & 0 & 0 & 11 \\
\hline
\end{tabular}
\end{center}

\section{Stack Instructions}
\begin{itemize}
    \item Rsrc2\_val is the stack pointer, as it's the address of the operation.
    \item ALUop's Inc2 and Dec2 are used to manipulate the stack pointer, thus the output of the ALU will be the new stack pointer.
    \item In case of Call, Rsrc1\_sel is none, as no register is used. It is the PC pushed at the memory.
    \item In case of Call, Rdst1\_sel, is the register holding the new address.
    \item In case of Ret and Rti, no registers are affected, as the PC is updated at the fetch stage.
    \item R/W memory (11) is write and (10) is read.
\end{itemize}

\begin{center}
    \captionof{table}{Stack Instruction Control Signals Part I\label{tab:stackop1}}
\begin{tabular}{||p{20mm}| p{15mm}| p{15mm}| p{15mm}| p{15mm}| p{15mm}| p{15mm}||} 
\hline
Instruction & OPCode & ALUop & Rsrc1 selector & Rsrc2 selector & Rdst1 selector & Rsrc2 val \\ [0.5ex] 
\hline\hline
PUSH & IB[31:25] & 1011 & 0 and IB[27:25] & 1111 & 1111 & SP(32 bits) \\
\hline
POP & IB[31:25] & 1010 & 1111 & 1111 & 0 and IB[27:25] & SP(32 bits) \\
\hline
CALL & IB[31:25] & 1011 & 1111 & 1111 & 0 and IB[27:25] & SP(32 bits) \\
\hline
RET & IB[31:25] & 1010 & 1111 & 1111 & 1111 & SP(32 bits) \\
\hline
RTI & IB[31:25] & 1100 & 1111 & 1111 & 1111 & SP(32 bits) \\
\hline
\end{tabular}
\end{center}

\begin{center}
    \captionof{table}{Stacks Instruction Control Signals Part II\label{tab:stackop2}}
\begin{tabular}{||p{20mm}| p{15mm}| p{15mm}| p{15mm}| p{15mm}| p{15mm}| p{15mm}||} 
\hline
Instruction & OP2 selector & Rdst2 (swap) & Branch /IO Enable & SP Enable & Branch Enable (JZ) & R/W Control Signal \\ [0.5ex] 
\hline\hline
PUSH & 1 & 1111 & 00 & 1 & 0 & 11 \\
\hline
POP & 1 & 1111 & 00 & 0 & 0 & 10 \\
\hline
CALL & 1 & 1111 & 00 & 0 & 0 & 11 \\
\hline
RET & 1 & 1111 & 00 & 0 & 0 & 10 \\
\hline
RTI & 1 & 1111 & 00 & 0 & 0 & 10 \\
\hline
\end{tabular}
\end{center}

\section{Jump Instructions}
\begin{itemize}
    \item Rsrc1\_sel is the address we are jumping to, that's why we need to verify that our prediction at the JZ case is correct.
    \item Branch/IO Enable is (11) as it is a branching instruction.
    \item Branch enable (1) to detect if the JZ operated correctly.
\end{itemize}

\begin{center}
    \captionof{table}{Jumpers Instruction Control Signals Part I\label{tab:jop1}}
\begin{tabular}{||p{20mm}| p{15mm}| p{15mm}| p{15mm}| p{15mm}| p{15mm}| p{15mm}||} 
\hline
Instruction & OPCode & ALUop & Rsrc1 selector & Rsrc2 selector & Rdst1 selector & Rsrc2 val \\ [0.5ex] 
\hline\hline
JMP & IB[31:25] & 0000 & 1111 & 1111 & 1111 & x \\
\hline
JZ & IB[31:25] & 0000 & 0 and IB[27:25] & 1111 & 1111 & x \\
\hline
\end{tabular}
\end{center}

\begin{center}
    \captionof{table}{Jumpers Instruction Control Signals Part II\label{tab:jop2}}
\begin{tabular}{||p{20mm}| p{15mm}| p{15mm}| p{15mm}| p{15mm}| p{15mm}| p{15mm}||} 
\hline
Instruction & OP2 selector & Rdst2 (swap) & Branch /IO Enable & SP Enable & Branch Enable (JZ) & R/W Control Signal  \\ [0.5ex] 
\hline\hline
JMP & x & 1111 & 11 & 0 & 0 & 00 \\
\hline
JZ & x & 1111 & 11 & 0 & 1 & 00 \\
\hline
\end{tabular}
\end{center}

\section{Special Instructions}
There's no interrupt instruction, but there's a bit called Interrupt, sent to the Control Unit as an input to indicate an interrupt signal was triggered.

\begin{center}
    \captionof{table}{Specials Instruction Control Signals Part I\label{tab:sop1}}
\begin{tabular}{||p{20mm}| p{15mm}| p{15mm}| p{15mm}| p{15mm}| p{15mm}| p{15mm}||} 
\hline
Instruction & OPCode & ALUop & Rsrc1 selector & Rsrc2 selector & Rdst1 selector & Rsrc2 val \\ [0.5ex] 
\hline\hline
NOP & IB[31:25] & 0000 & 1111 & 1111 & 1111 & x \\
\hline
SWAP & IB[31:25] & 0000 & 0 and IB[27:25] & 0 and IB[24:22] & 0 and IB[24:22] & x \\
\hline
Reset & IB[31:25] & 0000 & 1111 & 1111 & 1111 & x\\
\hline
Int & IB[31:25] & 0000 & 1111 & 1111 & 1111 & x \\
\hline
\end{tabular}
\end{center}

\begin{center}
    \captionof{table}{Specials Instruction Control Signals Part II\label{tab:sop2}}
\begin{tabular}{||p{20mm}| p{15mm}| p{15mm}| p{15mm}| p{15mm}| p{15mm}| p{15mm}||} 
\hline
Instruction & OP2 selector & Rdst2 (swap) & Branch /IO Enable & SP Enable & Branch Enable (JZ) & R/W Control Signals  \\ [0.5ex] 
\hline\hline
NOP & x & 1111 & 00 & 0 & 0 & 00 \\
\hline
SWAP & 0 & 0 and IB[27:25] & 00 & 0 & 0 & 11 \\
\hline
Reset & x & 1111 & 00 & 0 & 0 & 00 \\
\hline
Int & x & 1111 & 00 & 0 & 0 & 00 \\
\hline
\end{tabular}
\end{center}