\subsection{Intuition and Assumptions}
    Since data bus is $16-bits$ in width ie. $word$; so the following sizes are in terms of $words$.
    
    Since we are dealing with ONE main memory that contains all the Data and Instructions,
    and two caches one for Data and the other is for Instructions.

    This Design proposes to divide the Main memory into two parts, one for Data and one for Instructions,
    the upper most part is saved for Data from \textit{address} $00000000000$ to $01111111111$, while the Instructions
    \textit{address} starts from $10000000000$ to $11111111111$.

    Considering this, the $LSB$ of the address will indicate whther that address is corresponding to
    an instruction ($1$) or data ($0$).

    These assumptions allow us to:
    \begin{itemize}
        \item Limit collisions (conflicts) since for each Cache the tag size is reduced
            to only two bits since the actual address is 10 bits.
        \item Create an internal pipeline between the instruction cache and data cache,
            since the instruction cache is used only for reading, and filled periodically, more on 
            that on section \nameref{workflowSection}.
        \item No dirty bit array is used for instruction cache.
    \end{itemize}
     
\subsection{Caches Size}
    What is Given:
    \begin{itemize}
        \item Main memory: $4KB = 2^{12}B =2^{11} W$
        \item Cache Size: $512 Bytes = 256Words$
        \item Block Size: $16B = 8W$
        \item Number of Rows/Slots: $32$
        \item Number of Caches: $2$ one for Data and one for Instructions.
    \end{itemize}

    Sizes:
    \begin{itemize}
        \item Each Cache is $256Words$
        \item Tags are 2 bits in width.
        \item Validity is only 1 bit.
        \item Dirty bit is used only in case of Data cache.
        \item Extra bits required in total: $32*(2+1) + 16*(1) = 112bits$
    \end{itemize}


\subsection{Workflow}
\label{workflowSection}


