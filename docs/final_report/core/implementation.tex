\section{Completeness}
The processor is completely implemented except for \textbf{memory cache system}, which is already designed. However, we ran out of time trying to integrate it to our current implementation.

So, we reverted back to the two memories, \emph{Harvard} architecture. However, we included the cache system design in this report to share our thinking.

\section{Functionality}
The implemented processor, with \emph{Harvard} architecture, completely works with correct functionalities and hazard handling.

All instructions are executed successfully, except for:
\begin{itemize}
    \item \textbf{Interrupt} : which suffers from some implementation issues, as PC and CCR values aren't written correctly in data memory.
\end{itemize}

\section{General Notes}
\begin{itemize}
    \item Fetch stage outputs a single \emph{NOP} instruction, in case of \emph{Interrupt}, \emph{CALL}, \emph{JZ} and \emph{JMP}, due to the delay of reading data (PC values or branch address) from register file.
    \item Flushing happens only to the fetch stage, in case of incorrect predicted branch address for \emph{JZ} instruction.
    \item We used \emph{ghdl} as a \emph{VHDL} compiler and \emph{GTKWave} as a simulator throughout the development process for fast development and debugging. We included our development setup, along with the main deliverables of the project. 
\end{itemize}