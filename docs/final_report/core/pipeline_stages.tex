\section{Overview}
This section discusses the 5 stages of our system and their functionalities.

\subsection{Fetch Stage}
\begin{itemize}
    \item Responsible for fetching the next instruction.
    \item Can take two cycles in case of 32-bit instructions.
    \item Contains a branch prediction unit to determine the next address to be fetched in case of branching.
    \item Outputs the instruction bits into IF/ID Buffer.
    \item Reads from register file in the second half of cycle.
\end{itemize}

\subsection{Decode Stage}
\begin{itemize}
    \item Responsible for decoding the instruction bits into control signals.
    \item Outputs the corresponding signals to ID/EX Buffer.
    \item Contains register file to output operand values and register-related operations.
    \item Determines the correct branch address in case of branching instructions by using Branch Address Unit.
    \item Reads from register file in the second half of cycle.
    \item Reads IN port, in case of IN operation and propagates it to be written in Write-Back stage.
\end{itemize}

\subsection{Execute Stage}
\begin{itemize}
    \item Responsible for ALU operations.
    \item Determines the correct ALU output and pass it with other signals to EX/M Buffer.
    \item The ALU operations and CCR update are done in the first half of cycle (to avoid $JZ$ branch hazards).
\end{itemize}

\subsection{Memory Stage}
\begin{itemize}
    \item Responsible for Data Memory IO.
    \item Returns stored PC, in case of $RET$ and $RTI$.
    \item Stores PC and CCR, in case of interrupt.
\end{itemize}

\subsection{Write-Back Stage}
\begin{itemize}
    \item Responsible for passing correct output values to the destination registers.
    \item Write back is done in the first half of cycle.
\end{itemize}

\section{Intermediate Buffers}
Each buffer has internal latches. 
The buffer updates its latches when any input changes, regardless of clock. 
The buffer outputs the value of the internal latches at the \textbf{rising edge} of the clock.

Flush signal takes precedence over stall signal. That's it, if the buffer received both flush and stall, it must flush its internal buffers.
\subsection{IF/ID Buffer}
\subsubsection{Registers}
\begin{itemize}
    \item Instruction Register (32 bits)
    \item Next Address Register (32 bits)
    \item Incremented PC Register (32 bits)
    \item Hashed Address Register (4 bits)
    \item Interrupt Register (1 bit)
\end{itemize}

\subsubsection{Control Signals}
\begin{itemize}
    \item Flush: clear buffer (1 bit)
    \item Stall: freeze buffer (1 bit)
\end{itemize}

\subsection{ID/EX Buffer}
\subsubsection{Registers}
\begin{itemize}
    \item Operand Registers (2$X$32 bits)
    \item Destination Register (2$X$4 bits)
    \item Destination Register Value (32 bits)
    \item OpCode Register (7 bits)
    \item ALU Operation (4 bits)
    \item R/W Register (2 bits)
    \item Interrupt Register (1 bit)
\end{itemize}

\subsubsection{Control Signals}
\begin{itemize}
    \item Stall (IN): freeze buffer (1 bit)
    \item Destination Register (OUT) (4 bits)
\end{itemize}

\subsection{EX/M Buffer}
\subsubsection{Registers}
\begin{itemize}
    \item ALUout Register (32 bits)
    \item MEM IN Register (32 bits)
    \item Memory Address (32 bits)
    \item Opcode Register (7 bits)
    \item Destination Register (2$X$4 bits)
    \item Destination Register Value (2$X$32 bits)
    \item R/W Register (2 bits)
    \item Interrupt Register (1 bit)
\end{itemize}

\subsubsection{Control Signals}
\begin{itemize}
    \item Destination Register (OUT) (4 bits)
\end{itemize}

\subsection{M/WB Buffer}

\subsubsection{Registers}
\begin{itemize}
    \item ALUout (32 bits)
    \item MEM OUT (32 bits)
    \item OpCode (7 bits)
    \item Destination Register (2$X$4 bits)
    \item Destination Register Value (2$X$32 bits)
\end{itemize}


\section{Special Workflows}

\subsection{CALL Workflow}
\begin{itemize}
    \item Rdest value is loaded in fetch stage (like branches) and stored in PC.
    \item The current value of PC is propagated through the pipe, until it reaches the memory stage, where it's stored in data memory.
\end{itemize}

\subsection{RET Workflow}
\begin{itemize}
    \item Compiler inserts 3 NOPs after each RET instruction to avoid any hazards.
    \item Once the RET operation reaches the memory stage it loads the PC value from stack (like a normal POP) and uses PC Navigator to write it to the PC.
    \item \textbf{NOTE:} Data hazards related to SP are handled normally through hazard detection unit.
\end{itemize}

\subsection{Interrupt Workflow}
\begin{itemize}
    \item Interrupt signal is passed to the PC Control Unit and IF/ID Buffer, the fetch stage is stalled for two cycles to fetch the interrupt address and the Interrupt Bit propagates through the whole pipe, until it reaches the memory stage.
    \item In the memory stage, the interrupt stalls the pipe one cycles to be able to push both PC and CCR into stack.
\end{itemize}

\subsection{RTI Workflow}
\begin{itemize}
    \item Compiler inserts 3 NOPs after each RTI instruction to avoid any hazards.
    \item  Once the RTI operation reaches the memory stage it loads the PC value from stack (like a normal POP) and uses PC Navigator to write it to the PC.
    \item However, RTI stalls the pipe for one cycle to be able to load CCR, too.
    \item \textbf{NOTE:} Data hazards related to SP are handled normally through hazard detection unit.
\end{itemize}